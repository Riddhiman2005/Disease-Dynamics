\documentclass[a4paper]{article}
\usepackage{amsmath,amsthm,amssymb}
\usepackage{mathrsfs}
\usepackage{authblk}
\usepackage{graphicx}

\addtolength{\textwidth}{1cm}

\usepackage{amsmath}
\DeclareMathOperator\arctanh{arctanh}
\DeclareSymbolFontAlphabet{\mathrsfs}{rsfs}
\DeclareMathAlphabet{\mathcal}{OMS}{cmsy}{m}{n}
\newcommand{\scri}{\mathrsfs{I}} \newcommand{\izero}{\imath^0}
\newcommand{\const}{\text{const}} \newcommand{\be}{\begin{equation}}
\newcommand{\ee}{\end{equation}}
\providecommand{\todo}[1]{\textsl{\textbf{TODO:}#1}} 
\renewcommand{\figurename}{Figure}



\title{Numerical Investigation of Wave Solutions in the Teukolsky Equation for Schwarzschild de-Sitter Spacetime}
\author{AZ}
\date{June 18,2023}

\begin{abstract}
    Notes 
\end{abstract}
%%%%%%%%%%%%%%%%%%%%%%%%%%%%%%
\begin{document}

\maketitle
%%%%%%%%%%%%%%%%%%%%%%%%%%%%%%

%%%%%%%%%%%%%%%%%%%%%%%%%%%%%%
\section{Wave equation on de Sitter spacetime}
%%%%%%%%%%%%%%%%%%%%%%%%%%%%%%

\subsection{Treatment of the origin}

We want to solve the following equation for the unknown $u(r,t)$ using finite differences
\be\label{eq:spherical_wave} \partial_t u = \partial_r u + \frac{2}{r} u,\ee
on the domain $r\in[0,R]$. Spherical symmetry implies the following boundary condition at the origin 
\be\label{eq:spherical_bc} \partial_r u(0,t) = 0.\ee
One might be inclined to directly discretize Eq.~\eqref{eq:spherical_wave} and then impose the boundary condition Eq.~\eqref{eq:spherical_bc} in a one-sided derivative operator. It turns out that this approach is numerically unstable. Instead, we use the following identity
\[ \frac{u}{r} =  \partial_r u - r \, \partial_r \left(\frac{u}{r}\right), \]
to rewrite Eq.~\eqref{eq:spherical_wave} as
\[  \partial_t u = 3 \partial_r u - 2 r\, \partial_r \left( \frac{u}{r} \right). \]
Of course, the division by $r$ is still there, but now the discretization with the boundary condition \eqref{eq:spherical_bc} is stable.

%%%%%%%%%%%%%%%%%%%%%%%%%%%%%%
\section{Wave equation on Schwarzschild-de Sitter spacetime}
%%%%%%%%%%%%%%%%%%%%%%%%%%%%%%
The Schwarzschild-de Sitter (SdS) metric on the static patch
\begin{equation}\label{eq:ssds}
 ds^2 = - f dt^2 + \frac{1}{f} dr^2 + r^2 d\omega^2,\qquad f(r) = 1-\frac{r^2}{L^2} - \frac{2M}{r}\,.
\end{equation}
The metric is singular at the roots of $f$. Assuming $0<r_e<r_c$ and setting $r_0=-(r_e+r_c)$, we write
\be\label{eq:f} f = \frac{1}{L^2 r} (r-r_e)(r_c-r)(r-r_0), \quad L^2 = r_e^2 + r_e r_c + r_c^2.\ee

\subsection*{Hyperboloidal coordinates}
Introduce hyperboloidal time $\tau$ as usual with the height function $h(r)$ and boost $H(r)$
\[ \tau = t - h(r), \qquad H(r):= \frac{dh}{dr}.\]
The hyperboloidal SdS metric reads
\be\label{eq:metric_fH} ds^2 = - f d\tau^2 - 2 fH d\tau dr + \frac{1}{f}\left(1-f^2 H^2\right) dr^2 + r^2 d\omega^2. \ee
We use the freedom in $H$ to remove the singularity of the metric by requiring $1-f^2 H^2 \sim f$ near the roots of $f$. There are many choices available to achieve this. We need a choice that has good numerical properties. For example, the characteristic speeds should be reasonable.

The characteristic speeds of spherical light rays can be obtained via setting $ds^2=0$ and solving for $c_\pm = dr/d\tau$ which satisfies
\[ \frac{1}{f} (1-f^2 H^2) c_\pm^2 - 2 f H c_\pm - f = 0.\]
We get
\[ c_{\pm} = -\beta \pm \frac{\alpha}{\gamma} = \alpha^2 (fH\pm1) = \frac{f}{\mp 1+ fH}. \]
By construction, $c_+$ vanishes at the left boundary and $c_-$ vanishes at the right boundary. We also need $c_\pm$ to have ``reasonable'' finite values at their respective boundaries when they do not vanish.

{\bf Choice 1:}\\
\be\label{eq:fH} f H = 2 \frac{r-r_e}{r_c-r_e} - 1.\ee
We get the metric
\[ ds^2 = - f dt^2 - 2 \left(2 \frac{r-r_e}{r_c-r_e} - 1\right) d\tau dr + \frac{4 L^2 r}{(r_c-r_e)^2 (r-r_0)} dr^2 + r^2 d\omega^2. \]
Now using \eqref{eq:f} and \eqref{eq:fH}
\[ c_{+} = \frac{1}{2L^2 r} (r-r_e)(r-r_0)(r_c-r_e), \qquad  c_{-} = \frac{1}{2L^2 r} (r_c-r)(r-r_0)(r_c-r_e) \]
As expected, $c_+(r_e)=0=c_-(r_c)$. When they don't vanish at the boundaries, we have
\[ c_+(r_c) = \frac{(r_c-r_e)^2 (r_e+2 r_c)}{2L^2 r_c}, \qquad c_-(r_e) =  \frac{(r_c-r_e)^2 (r_c+2 r_e)}{2L^2 r_e}. \]
We are interested in the large $r_c$ case. We see that $c_(r_e) \sim r_c$. This choice is not good because the ingoing characteristic near the black hole horizon increases with large $r_c$, overly restricting our CFL condition. 

{\bf Choice 2:}\\
We write our previous choice as
\[ f H = \frac{r_c-r}{r_c-r_e} - \frac{r-r_e}{r_c-r_e},\]
and modify it slightly as
\[ f H = \frac{r_e}{r} \frac{r_c-r}{r_c-r_e} - \frac{r-r_e}{r_c-r_e}.\]
The characteristics read now
\[ c_{+} = \frac{1}{L^2 (r+r_e)} (r-r_e)(r-r_0)(r_c-r_e), \qquad  c_{-} = \frac{1}{L^2 (r+r_c)} (r_c-r)(r-r_0)(r_c-r_e) \]
The non-vanishing boundary speeds are
\[ c_+(r_c) = \frac{(r_c-r_e)^2 (r_e+2 r_c)}{L^2 (r_c+r_e)}, \qquad c_-(r_e) =  \frac{(r_c-r_e)^2 (r_c+2 r_e)}{L^2 (r_c+r_e)}. \]
Both speeds are on the order of unity for large $r_c$. The small modification fixes the behavior of the characteristic speeds.

{\bf Choice 3:}\\
Another choice is the one by Hintz and Xie in \cite{hintz:2021}. They chose the height function as
\[ -h(r) = \frac{1}{2\kappa_e} \ln(r-r_e) + \frac{1}{2\kappa_c} \ln(r-r_c). \]
So the boost is then
\[ -H = \frac{1}{2\kappa_e (r-r_e)}+ \frac{1}{2\kappa_c (r-r_c)}.\]
In particular
\[ fH = - \frac{r_e}{r} \frac{r_c-r}{r_c-r_e} \frac{r-r_0}{r_e-r_0} +  \frac{r_c}{r} \frac{r-r_e}{r_c-r_e} \frac{r-r_0}{r_c-r_0}. \]
We get the metric
\[ ds^2 = - f dt^2 - 2 \left(2 \frac{r-r_e}{r_c-r_e} - 1\right) d\tau dr + \frac{4 \ell^2 r}{(r_c-r_e)^2 (r-r_0)} dr^2 + r^2 d\omega^2. \]

Now using \eqref{eq:f} and \eqref{eq:fH}
\[ c_{+} = \frac{1}{2\ell^2 r} (r-r_e)(r-r_0)(r_c-r_e), \qquad  c_{-} = \frac{1}{2\ell^2 r} (r_c-r)(r-r_0)(r_c-r_e) \]

{\bf Choice 4:}\\
Take the tortoise coordinate defined through
\[ r_\ast = \int\frac{1}{f}dr \]
The metric becomes 
\[ ds^2 = f \left( -dt^2 + dr_\ast^2\right) + r(r_\ast)^2 d\omega^2. \]
Define the new time coordinate as 
\[ \tau = t - \sqrt{1+r_\ast^2}. \]
The main advantage of this construction is that it's easy to adapt to the requirements of the numerical computation as follows
\[ \tau = t - \sqrt{K^2+(r_\ast-p)^2}. \]
For now, we just set $p=0$ and recompactify space using
\[ r_\ast  = \frac{\rho_\ast}{\Omega} \quad \rm{with} \quad \Omega = \frac{1-\rho_\ast^2}{2}.\]
This transformation maps the radial coordinate $r_\ast\in(-\infty,\infty)$ to $\rho\in[-1,1]$. The metric reads then
\[  ds^2 =  \frac{1}{\Omega^2} \left\{ f\left(-\Omega^2 d\tau^2 - 2 \rho_\ast d\tau d\rho_\ast +  d\rho_\ast^2\right) + \rho^2 d\omega^2 \right\}, \] 
where we have defined $\rho :=  \Omega r$. Note that $\rho$ has the same domain and limits as $\rho_\ast$. This metric is regular, so the transformed equation will be regular as well.

%%%%%%%%%%%%%%%%%%%%%%%%%%%%%%
\subsection*{Scalar wave equation}
We consider the scalar wave equation
\[ \Box \psi = \frac{1}{\sqrt{-g}} \partial_\mu \left(\sqrt{-g} g^{\mu\nu} \partial_\nu \psi \right) = 0  \]
After decomposing in spherical modes and writing out
\[ \partial_{{t}}^{\,2}\psi  = f^2 \partial_r^2 \psi + f \left(\frac{2f}{r}+f'\right) \partial_r \psi - \frac{f \ell^2}{r^2}. \]
We rescale by $r$ via $u:=\psi/r$ to get
\[ \partial_{{t}}^{\,2}u  = f^2 \partial_r^2 u + f f' \partial_r u - \frac{f}{r^2} (r f' + \ell^2). \]
Transforming into the tortoise coordinate gives us
\[ \partial_{{t}}^{\,2}u  = \partial_{r_\ast}^2 u  - \frac{f}{r^2} (r f' + \ell^2). \]
Now perform the hyperboloidal tranformation
\[ \tau = t-\sqrt{1+r_\ast^2} \]
in combination with compactification
\[ r_\ast = \frac{\rho_\ast}{\Omega} \quad {\rm with} \quad \Omega=\frac{1-\rho_\ast^2}{2} \]
The hyperboloidal transformation reads in compactifying coordinates
\[ \tau = t - \frac{1+\rho_\ast^2}{1-\rho_\ast^2}\]
The derivative operators transform as
\[ \partial_\tau = \partial_t, \qquad \partial_{r_\ast} = \frac{2}{1+\rho_\ast^2} \left(-\rho_\ast \partial_\tau + \Omega^2 \partial_{\rho_\ast}\right)  \]
The resulting equation reads then
\[ -\partial_\tau^2 - 2 \rho_\ast \partial_\tau \partial_{\rho_\ast} + \Omega^2 \partial_{\rho_\ast}^2 
- \frac{\Omega}{1+\rho_\ast^2}\left(2 \partial_\tau + \rho_\ast (3+\rho_\ast^2) \partial_{\rho_\ast}\right)=
\frac{f (1+\rho_\ast^2)}{4\Omega^2 r^2}(r f'+\ell^2). \]
Note that $\Omega^2 r^2$ is regular at both infinities, so we can define $\rho = \Omega^2 r^2$ which lives on the same domain as $\rho_\ast\in[-1,1]$.

%%%%%%%%%%%%%%%%%%%%%%%%%%%%%%
%%%%%%%%%%%%%%%%%%%%%%%%%%%%%%
\newpage
\begin{thebibliography}{}
\bibitem{Bizon} Bizoń, Piotr, Tadeusz Chmaj, and Patryk Mach. "A toy model of hyperboloidal approach to quasinormal modes." arXiv preprint arXiv:2002.01770 (2020).
\bibitem{carter:1968} B.~Carter, Hamilton-Jacobi and Schr\"{o}dinger separable solutions of Einstein's equations, Commun.~Math.~Phys.~{\bf 10}, 280 (1968).
\bibitem{Dias12} O.J.C.~Dias, J.E.~Santos, and M.~Stein, Kerr-AdS and its Near-horizon Geometry: Perturbations and the Kerr/CFT Correspondence, J. High Energy Phys. (2012) 2012: 182, arXiv:1208.3322 [hep-th]. 
\bibitem{Bini12} Bini, D., Esposito, G. and Geralico, A., 2012. de Sitter spacetime: effects of metric perturbations on geodesic motion. General Relativity and Gravitation, 44(2), pp.467-490.
\bibitem{brady:1999} Brady, P. R., Chambers, C. M., Laarakkers, W. G., and Poisson, E. Radiative falloff in Schwarzschild–de Sitter spacetime. Physical Review D, 60(6), 064003 (1999).
\bibitem{hintz:2021} Hintz, P., and Xie, YQ. Quasinormal modes of small Schwarzschild-de Sitter black holes. arXiv:2105.02347 (2021).
\bibitem{zenginoglu:2011} A.~Zengino\u{g}lu and G.~Khanna, Null infinity waveforms from extreme-mass-ratio inspirals in Kerr spacetime, Phys. Rev. X {\bf 1}, 021017 (2011), arXiv:1108.1816  [gr-qc].
\bibitem{zenginoglu_tiglio_2009} Zengino\u{g}lu, A., and Tiglio, M. Spacelike matching to null infinity. Physical Review D, 80(2), 024044 (2009). 
\end{thebibliography}
%%%%%%%%%%%%%%%%%%%%%%%%%%%%%%




\end{document}
